% !TEX root = ../main.tex
\documentclass[../main.tex]{subfiles}

\chapter{Aprendizaje con Kernels}

\begin{quotation}

Alza del hiperparametro origina nueva alza del hiperparametro \\
Alza de los errores \\
Provoca instantáneamente la duplicación de los errores \\
Alza de las métricas \\
Origina alza de las métricas.
\NC{Poner algo profundo por el estilo. (?)}

\end{quotation}

\section{Introducción}

Un kernel es una función simétrica y definida positiva $k(\cdot,\cdot)$ que
puede ser entendida como una medida de similitud entre los argumentos que opera.
En el siguiente capítulo, se definen estos objetos matemáticos de manera formal,
se investigan sus características y se derivan algunos métodos del aprendizaje
de máquinas que toman ventaja sus propiedades.

\section{Terminología y propiedades}

El término \textbf{kernel} proviene del estudio de operadores integrales en el
campo del análisis funcional. En tal contexto se les identifica como aquellas
funciones $k$ que determinan un operador $T_k$ a través de:
\begin{equation}
    (T_k f) (x) = \int_{\mathcal{X}} k(x,x') f(x') dx
\end{equation}

En concordancia con la perspectiva que se desea abarcar, se denotará como kernel
a toda función $k: \mathcal{X} \times \mathcal{X} \rightarrow \mathbb{R}$ en el
espacio de características \footnote{El codominio del kernel $k$ no tiene por que
restringirse a los reales, para ciertas aplicaciones puede ser conviene utilizar
los complejos. En general las propiedades de interés se preservan en ambos cuerpos,
por lo que por simplicidad se considera solo el caso real en esta monografía.}.
Dentro de tal clase de funciones, son de importancia a aquellas capaces de ``generalizar"
el concepto de \textit{porducto interno}, el \index{Teorema de Mercer}
\textbf{Teorema de Mercer} permite identificar tal subconjunto.
\NC{Agregar apendice sobre Teoria de Medida + integrar respecto a una medida
learning with Kernels}
\begin{theorem}[Teorema de Mercer]
\label{mercer}
Sea $(\mathcal{X},\mu)$ un espacio de medida finita y $k \in L_{\infty}(\mathcal{X}^2)$
una función real y simétrica, tal que el operador integral:

\begin{gather}
    \begin{aligned}
        T_k &: L_{2}(\mathcal{X}) \rightarrow L_{2}(\mathcal{X}) \\
        (T_k f) (x) &:= \int_{\mathcal{X}} k(x,x') f(x') d\mu(x')
    \end{aligned}
\end{gather}

es definido positivo, es decir, para toda $f \in L_{2}(\mathcal{X})$ se cumple:
\begin{equation}
    \int_{\mathcal{X}^2} k(x,x') f(x) f(x') d\mu(x) d\mu(x') \geq 0
\end{equation}

Si $\psi_j \in L_2(\mathcal{X})$ son las funciones propias ortonormales de $T_f$
asociadas a los valores propios $\lambda_j > 0$, ordenados de manera decreciente.
Entonces,
\begin{enumerate}
    \item $(\lambda_j)_j \in l_1$
    \item $k(x,x')=\sum_{j=1}^{N_{\mathcal{H}}} \lambda_j \psi_j(x)\psi_j(x')$
    se cumple para casi todos los elementos $x,x' \in \mathcal{X}$. Además $N_{\mathcal{H}} \in \mathbb{N}$
    o $N_{\mathcal{H}} = \infty$, en este último caso, la serie respectiva converge
    absoluta y uniformemente para casi todos los elementos $x,x' \in \mathcal{X}$.
\end{enumerate}

\end{theorem}
\NC{Esquema de la demostración ?}
De la segunda implicación del teorema anterior, se puede deducir que
$k(x,x')$ corresponde a un producto interno en $l_2^{N_{\mathcal{H}}}$ definido
a través de $k(x,x')=\langle \Phi(x),\Phi(x') \rangle$ donde:
\begin{gather*}
    \begin{aligned}
        \Phi: \mathcal{X} ~ &\rightarrow ~ l_2^{N_{\mathcal{H}}} \\
        x ~ &\mapsto ~ (\sqrt{\lambda_j} \psi_j(x))_{j=1,\ldots,N_{\mathcal{H}}}
    \end{aligned}
\end{gather*}

Para casi todo $x \in \mathcal{X}$. En este caso, $\Phi$ se interpreta como una
aplicación al espacio de características. Vale destacar, que tal espacio posee
producto interno, más aún, el hecho de que exista convergencia uniforme en
la serie, significa que para cualquier precisión $\varepsilon >0$, debe existir un
$n \in \mathbb{N}$ tal que $k$ puede ser aproximado como un producto interno en
$\mathbb{R}^{n}$. Lo anterior se observa al notar que para casi todo $x,x' \in \mathcal{X}$,
se tiene $|k(x,x') - \langle \Phi^{n}(x), \Phi^{n}(x') \rangle| < \varepsilon$,
donde $ \Phi^{n}(x): x \mapsto (\sqrt{\lambda_1}\psi_1(x), \ldots,\sqrt{\lambda_n}\psi_n(x))$.
En tal contexto, se puede interpretar al espacio de características como un espacio
finito dimensional dentro de cierta precisión $\varepsilon$. Tal observación se
concreta en el siguiente teorema:

\begin{theorem}[Aplicación kernel de Mercer]
\label{mercer_app}
Si $k$ es un kernel que cumple las condiciones del teorema (\ref{mercer}), se
puede construir una aplicación $\Phi$ a un espacio donde $k$ se comporta
como un producto interno:
\begin{equation*}
    k(x,x') = \langle \Phi(x), \Phi(x') \rangle
\end{equation*}
para casi todo $x,x' \in \mathcal{X}$. Más aún, para todo $\varepsilon >0$, existe
una aplicación $\Phi_{n}$ a un espacio $n-$dimensional con producto interno tal que
\begin{equation*}
    |k(x,x') - \langle \Phi_{n}(x), \Phi_{n}(x') \rangle| < \varepsilon
\end{equation*}
para casi todo $x,x' \in \mathcal{X}$.
\end{theorem}

La construcción que ofrece el \textbf{teorema de Mercer} permite bajo ciertas
condiciones, obtener una representación del espacio inicial $\mathcal{X}$ en un espacio
con producto interno de alta dimensionalidad (dentro de cierta precisión $\varepsilon$).
Esta construcción se obtiene a través de una descomposición espectral del operador
$T_k$ asociado al kernel $k$.

Al observar la representación del kernel en función de la descomposición espectral
de $T_k$ y reemplazar en la definición de positividad de este último, se obtiene:
\begin{gather}
    \begin{aligned}
        \int_{\mathcal{X}^2} k(x,x') f(x) f(x') d\mu(x) d\mu(x') =
        \sum_{j}^{N_{\mathcal{H}}}\lambda_j \left[\int_{\mathcal{X}}\psi_j(x)f(x)d\mu(x)\right]^2 \geq 0 \\
        ~~~ \forall f \in L_{2}(\mathcal{X})
    \end{aligned}
\end{gather}





Debido a las propiedades geometricas y resultados de aproximación que ofrecen
los espacios con producto interno,


A Reproducing Kernel Hilbert Space (RKHS) is first of all a Hilbert
space, that is, the most natural extension of the mathematical model
for the actual space where everyday life takes place (the Euclidean space
JR3) . When studying elements of some abstract set S it is convenient
to consider them as elements of some other set S' on which is already
defined a structure relevant to the problem to be treated . It can be for
instance an order structure, a vector structure, a metric structure or
a mixing of algebraic and topological structures. For this we need an
"imbedding theorem" or a "representation theorem" . Through this kind
of theorem the study of elements of S is transferred to their "represen-
ters" in S' and can be carried out using the structure on S' . For their
richness and simplicity Hilbert spaces are introduced as often as possible
when a vector structure and an inner product can be exploited. They
provide powerful mathematical tools and geometric concepts on which
our intuition can rest. The phrase "RKHS method" is generic to nam e a
method based on the embedding of the abstract set S into some RKHS
S'
